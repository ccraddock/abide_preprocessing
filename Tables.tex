\section{Tables (optional)}

\textit{Tables supporting the Data Descriptor.  These can provide summary information (sample numbers, demographics, etc.), but they should generally not be used to present primary data (i.e. measurements). Tables containing primary data should be submitted to an appropriate data repository. }

\textit{Tables may be provided within the Word document or as separate files (tab-delimited text or excel files).  Legends, where needed, should be included here in the Word document. Generally, a Data Descriptor should have fewer than ten Tables, but more may be allowed when needed. Tables may be of any size, but only Tables which fit onto a single printed page will be included in the PDF version of the article (up to a maximum of three).   }

\textit{Authors are required to provide detailed information accounting for 1) each sample or subject employed in the study, 2) the data-generating assays applied to each sample, and 3) the resulting data outputs. These should be detailed enough to identify measurements that were derived from common samples, including replicates measurements, or measurements of different kinds derived from a common sample (e.g. parallel proteomic and transcriptomic datasets). Authors may choose to submit this detailed information using the ISA-tab experimental metadata format, but any information that is essential to readers of the Data Descriptor should also be formatted as human-readable Tables. }

\textit{Of course, where human data is involved, we recognize that privacy controls may preclude highly detailed descriptions of patients or participants. Please make sure that any privacy-related limitations on data sharing are discussed in the cover letter of your submission.  }
