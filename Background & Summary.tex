\section{Background & Summary} 
\textit{700 words maximum

An overview of the study design, the assay(s) performed, and the created data, including any background information needed to put this study in the context of previous work and the literature. The section should also briefly outline the broader goals that motivated the creation of this dataset and the potential reuse value.  This section should include citations to the literature as needed.  We also encourage authors to include a figure that provides a schematic overview of the study and assay(s) design.} 


 Grass-roots initiatives such as the 1000 Functional Connectomes Project (FCP)
		    and International Neuroimaging Data-sharing Initiative (INDI) (\href{http://fcon_1000.projects.nitrc.org}{fcon\_1000.projects.nitrc.org}) are
successfully amassing and sharing large-scale brain imaging datasets, with the
goal of recruiting the broader scientific community into the fold of
neuroimaging research. Unfortunately, despite the increasing breadth and scale
of openly available data, the vast domain-specific knowledge and computational
resources necessary to derive scientifically meaningful information from
unprocessed neuroimaging data has limited their accessibility. The Neuro Bureau
Preprocessing Initiative [2] has taken on this challenge,
generating and openly sharing preprocessed data and common derivatives for the
large-scale ADHD-200 dataset [3,7]. This initiative has grown to
include preprocessed DTI data and derivatives for 180 healthy individuals from
INDI's Beijing Enhanced Sample [4]. The next planned release will
include resting state and structural data from the 1,112 subject Autism Brain
Imaging Data Exchange (ABIDE) dataset [5].