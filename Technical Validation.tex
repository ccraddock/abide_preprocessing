\section{Technical Validation}

This section presents any experiments or analyses that are needed to support the technical quality of the dataset.  This section may be supported by up figures and tables, as needed. This is a required section; authors must present information justifying the reliability of their data. 

Possible content may include:
\begin{itemize}
\item experiments that support or validate the data collection procedure (e.g. negative controls, or an analysis of standards to confirm measurement linearity)
\item statistical analyses of experimental error and variation
\item phenotypic or genotypic assessments of biological samples (e.g. confirming disease status, cell line identity, or the success of perturbations)
\item general discussions of any procedures used to ensure reliable and unbiased data production, such as blinding and randomization, sample tracking systems, etc. 
\item any other information needed for the peer reviewers to assess technical rigor
\item Generally, this should not include:
\item Follow-up experiments aimed at testing or supporting an interpretation of the data
\item Statistical hypothesis testing (e.g. tests of statistical significance, identifying differentially expressed genes, trend analysis, etc)
\item Exploratory computational analyses like clustering and annotation enrichment (e.g. GO analysis). 
\end{itemize}
